
\begin{thebibliography}{10}
	\bibitem{IoT}
		総務省:IoTデバイスの急速な普及 ,情報通信白書(オンライン),入手先\textless http://www.soumu.go.jp/johotsusintokei/whitepaper/ja/h30/html/nd111200.html\textgreater(参照2018-06-17).
    \bibitem{Mirai}
        宮田健:IoTデバイスを狙うマルウェア\\「Mirai」とは何か――その正体と対策,\\Tech Factry(オンライン),入手先\textless http:\slash\slash{}techfactory.itmedia.co.jp\slash{}tf\slash{}articles\slash{}1704\slash{}13\slash{}news010.html\textgreater\\(参照2018-06-20).
    \bibitem{Dyn}
       Scott Hilton:Dyn Analysis Summary Of Friday October 21 Attack,Oracle Dyn(オンライン),入手先\textless https:\slash\slash{}dyn.com\slash{}blog\slash{}dyn-analysis-summary-of-friday-october-21-attack\textgreater (参照2018-06-20).

	\bibitem{攻撃性能評価を行った研究}
    	長柄啓吾,松原豊,青木克憲 ほか:組込みシステム向けマルウェアMiraiの攻撃性能評価,研究報告システム・アーキテクチャ,vol.2017-ARC-225,No.41,p1-6(2017)
    \bibitem{アノマリ手法}
        坂野加奈,上原哲太郎:アノマリ検知手法を用いたIoT機器のマルウェア感染検出,研究報告セキュリティ心理学とトラスト,vol.2018-SRT-27 No.3,p1-6 (2018)
    \bibitem{API}
     	青木一樹,後藤滋樹:マルウェア検知のためのAPIコールパターンの分析,電子情報通信学会総合大会講演論文集 2014年 情報・システム,vol.2,No.179,2014-03-04
     \bibitem{mirai}
     Jerry Gamblin:jgamblin/Mirai-Source-\\Code,GitHub(オンライン),入手先\textless https://github.com/jgamblin/Mirai-Source-Code\textgreater(参照2018-09-20)
\end{thebibliography}

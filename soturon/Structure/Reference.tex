
\begin{thebibliography}{30}
	\bibitem{IoT}
		総務省:IoTデバイスの急速な普及 ,情報通信白書(オンライン),\\入手先〈http://www.soumu.go.jp/johotsusintokei/whitepaper/ja/h30/html/ \\nd111200.html〉(参照2018-06-17)
    
    \bibitem{Mirai}
        宮田健:IoTデバイスを狙うマルウェア「Mirai」とは何か――その正体と対策,Tech Factry(オンライン),入手先〈http:\slash\slash{}techfactory.itmedia.co.jp\slash{}tf\slash{}articles\slash{}1704\slash{}13\slash{}news010.html〉(参照2018-06-20)

    \bibitem{Dyn}
       Scott Hilton:Dyn Analysis Summary Of Friday October 21 Attack,Oracle Dyn(オンライン),入手先〈https:\slash\slash{}dyn.com\slash{}blog\slash{}dyn-analysis-summary-of-friday-october-21-attack〉 (参照2018-06-20)
       
    \bibitem{code}
       Jerry Gamblin:jgamblin/Mirai-Source-Code,GitHub,(オンライン),入手先〈https://github.com/jgamblin/Mirai-Source-Code〉(参照2018-09-20)
    
    \bibitem{Wicked}
       鈴木聖子:IoTデバイスの脆弱性を突くマルウェア「Wicked」、Miraiの新手の亜種,ITmedia エンタープライズ,(オンライン)入手先〈http:\slash\slash{}techfactory.itmedia.co.jp\slash{}tf\slash{}articles\slash{}1704\slash{}13\slash{}news010.html〉\\(参照2018-06-20)
   
    \bibitem{Satori}
       @IT:マルウェア「Satori」による攻撃を国内初観測、従来のファイアウォール機能では対応が難しい?,@IT,(オンライン)入手先〈http:\slash\slash{}www.atmarkit.co.jp\slash{}ait\slash{}articles\slash{}1806\slash{}27\slash{}news083.html〉 (参照2018-09-23)

    \bibitem{Okiru}
       Nick Lewis:Mirai亜種のIoTマルウェア「Okiru」とは? 標的は「ARCプロセッサ」,Tech Factry(オンライン),入手先〈http:\slash\slash{}techfactory.itmedia.co.jp\slash{}tf\slash{}articles\slash{}1704\slash{}13\slash{}news010.html〉\\(参照2018-06-20)

    \bibitem{newMirai}
       岩崎 宰守:IoTマルウェア「Mirai」をWindowsから拡散、2017年に入って500システムへの攻撃を確認 -INTERNET Watch,入手先〈https:\slash\slash{}internet.watch.impress.co.jp\slash{}docs\slash{}news1046239.html〉(参照2018-012-26)
    
    \bibitem{国立}
        国立研究開発法人情報通信研究機構:日本国内でインターネットに接続されたIoT機器等に関する事前調査の実施について, NICT-情報通信研究機構(オンライン),入手先〈https:\slash\slash{}dyn.com\slash{}blog\slash{}dyn-analysis-summary-of-friday-october-21-
        attack〉 (参照2018-06-20)
        
    \bibitem{実践}
        青木一史,秋山満照,幾世知範 ほか:実践サイバーセキュリティモニタリング,pp103-105,コロナ社(2016)
    

        
    %\bibitem{github}
     %   GitHub, Inc:GitHub ,(オンライン)入手先\textless https:\slash\slash{}github.com\slash{}%〉\\(参照2018-09-20)
    
    \bibitem{攻撃性能評価を行った研究}
        長柄啓吾,松原豊,青木克憲 ほか:組込みシステム向けマルウェアMiraiの攻撃性能評価,研究報告システム・アーキテクチャ,vol.2017-ARC-225,No.41,p1-6(2017)
    
    \bibitem{観測と分析}
        鉄穎,楊笛,松本勉 ほか:IoTマルウェアによるDDoS攻撃の動的解析による観測と分析,情報処理学会論文誌,vol.59 No.5,p1321-1333(2018)

    \bibitem{パターン}
        佐藤純子,花田真樹,面和成 ほか:API呼び出しとそれに伴う経過時間とシステム負荷を用いた検知手法,コンピュータセキュリティシンポジウム2017論文集,vol.2017 No.2 (2017)

    \bibitem{挙動の差異}
        実行毎の挙動の差異に基づくマルウェア検知手法,上原哲太郎:アノマリ検知手法を用いたIoT機器のマルウェア感染検出,研究報告セキュリティ心理学とトラスト,vol.2018-SRT-27 No.3,p1-6 (2018)

    \bibitem{アノマリ}
        坂野加奈,上原哲太郎:アノマリ検知手法を用いたIoT機器のマルウェア感染検出,研究報告セキュリティ心理学とトラスト,vol.2018-SRT-27 No.3,p1-6 (2018)

    \bibitem{Arbor}
         Arbor:Arbor Networks Peakflow® 7.0 が、 DDoS 攻撃検知とミティゲーションの時間を大幅に短縮,入手先〈https:\slash\slash{}jp.arbornetworks.com\slash{}lorem-post-3\slash{}〉 (参照2018-12-22).
    
    \bibitem{ar}
         ビジネスネットワーク:アーバーネットワークス Arbor Peakflow SP/TMS あらゆる企業がDDoS攻撃で狙われる時代 No.1ベンダーの答えは「上流で止めろ!」,入手先,〈https:\slash\slash{}businessnetwork.jp\slash{}Portals\slash{}0\slash{}SP\slash{}1210\_arbor\slash{}?prtext〉 (参照2019-1-21).

    \bibitem{ClamAV}
         Nicholas Herbert,Melissa Taylor,Nicolette Verbeck:ClamAV,(オンライン),入手先〈https:\slash\slash{}www.clamav.Networks〉(参照2018-12-22).
    
    \bibitem{Linux}
       農林水産研究情報総合センター:UNIX(Linux)でよく使うコマンド ,(オンライン),入手先〈https:\slash\slash{}itcweb.cc.affrc.go.jp\slash{}affrit/doku.php?id=faq\slash{}tips\slash{}unix-command〉(参照2019-1-14)
    
    \bibitem{Cowrie}
        Micheloosterhof:Cowrie,(オンライン),入手先〈https:\slash\slash{}itcweb.cc.affrc.go.jp\slash{}affrit\slash{}doku.php?id=faq\slash{}tips/unix-command〉(参照2019-1-4)
    
    \bibitem{Virus}
        Google Inc:Virus Total,(オンライン),入手先〈https:\slash\slash{}www.virustotal.com\slash{}home\slash{}upload〉(参照2019-01-14)
\end{thebibliography}

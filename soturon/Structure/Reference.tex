
\begin{thebibliography}{30}
	\bibitem{IoT}
		総務省:IoTデバイスの急速な普及 ,情報通信白書(オンライン),入手先\textless http://www.soumu.go.jp/johotsusintokei/whitepaper/ja/h30/html/nd111200.html\textgreater(参照2018-06-17).
   .
    \bibitem{Dyn}
       Scott Hilton:Dyn Analysis Summary Of Friday October 21 Attack,Oracle Dyn(オンライン),入手先\textless https:\slash\slash{}dyn.com\slash{}blog\slash{}dyn-analysis-summary-of-friday-october-21-attack\textgreater (参照2018-06-20).
    
    \bibitem{newMirai}
       岩崎 宰守:IoTマルウェア「Mirai」をWindowsから拡散、2017年に入って500システムへの攻撃を確認 -INTERNET Watch,入手先\textless https:\slash\slash{}internet.watch.impress.co.jp\slash{}docs\slash{}news1046239.html\textgreater\\(参照2018-012-26).
    
    \bibitem{国立}
        国立研究開発法人情報通信研究機構:日本国内でインターネットに接続されたIoT機器等に関する事前調査の実施について, NICT-情報通信研究機構(オンライン),入手先 \textless https:\slash\slash{}dyn.com\slash{}blog\slash{}dyn-analysis-summary-of-friday-october-21-
        attack\textgreater (参照2018-06-20).
        
    \bibitem{Mirai}
        宮田健:IoTデバイスを狙うマルウェア\\「Mirai」とは何か――その正体と対策,\\Tech Factry(オンライン),入手先\textless http:\slash\slash{}techfactory.itmedia.co.jp\slash{}tf\slash{}articles\slash{}1704\slash{}13\slash{}news010.html\textgreater\\(参照2018-06-20)

    \bibitem{Owari}
        宮田健:IoTデバイスを狙うマルウェア\\「Mirai」とは何か――その正体と対策,\\Tech Factry(オンライン),入手先\textless http:\slash\slash{}techfactory.itmedia.co.jp\slash{}tf\slash{}articles\slash{}1704\slash{}13\slash{}news010.html\textgreater\\(参照2018-06-20)
    
    \bibitem{Satori}
        宮田健:IoTデバイスを狙うマルウェア\\「Mirai」とは何か――その正体と対策,\\Tech Factry(オンライン),入手先\textless http:\slash\slash{}techfactory.itmedia.co.jp\slash{}tf\slash{}articles\slash{}1704\slash{}13\slash{}news010.html\textgreater\\(参照2018-06-20)

    \bibitem{Okiru}
        宮田健:IoTデバイスを狙うマルウェア\\「Mirai」とは何か――その正体と対策,\\Tech Factry(オンライン),入手先\textless http:\slash\slash{}techfactory.itmedia.co.jp\slash{}tf\slash{}articles\slash{}1704\slash{}13\slash{}news010.html\textgreater\\(参照2018-06-20)
    \bibitem{攻撃性能評価を行った研究}
        長柄啓吾,松原豊,青木克憲 ほか:組込みシステム向けマルウェアMiraiの攻撃性能評価,研究報告システム・アーキテクチャ,vol.2017-ARC-225,No.41,p1-6(2017)
    
    \bibitem{観測と分析}
        鉄穎,楊笛,松本勉 ほか:IoTマルウェアによるDDoS攻撃の動的解析による観測と分析,情報処理学会論文誌,vol.59 No.5,p1321-1333(2018)
    \bibitem{パターン}
        佐藤純子,花田真樹,面和成 ほか:API呼び出しとそれに伴う経過時間とシステム負荷を用いた検知手法,コンピュータセキュリティシンポジウム2017論文集,vol.2017 No.2 (2017)
    \bibitem{挙動の差異}
        実行毎の挙動の差異に基づくマルウェア検知手法,上原哲太郎:アノマリ検知手法を用いたIoT機器のマルウェア感染検出,研究報告セキュリティ心理学とトラスト,vol.2018-SRT-27 No.3,p1-6 (2018)
    \bibitem{アノマリ}
        坂野加奈,上原哲太郎:アノマリ検知手法を用いたIoT機器のマルウェア感染検出,研究報告セキュリティ心理学とトラスト,vol.2018-SRT-27 No.3,p1-6 (2018)
    \bibitem{API}
         青木一樹,後藤滋樹:マルウェア検知のためのAPIコールパターンの分析,電子情報通信学会総合大会講演論文集 2014年 情報・システム,vol.2,No.179,2014-03-04
    \bibitem{Arbor}
         Arbor:Arbor Networks Peakflow® 7.0 が、 DDoS 攻撃検知とミティゲーションの時間を大幅に短縮,入手先 \textless https:\slash\slash{}jp.arbornetworks.com\slash{}lorem-post-3\slash{}\textgreater (参照2018-12-22).
    \bibitem{ClamAV}
         ClamAV (オンライン),入手先 \textless https:\slash\slash{}www.clamav.Networks\textgreater(参照2018-12-22).

    \bibitem{code}
     Jerry Gamblin:jgamblin/Mirai-Source-\\Code,GitHub(オンライン),入手先\textless https://github.com/jgamblin/Mirai-Source-Code\textgreater(参照2018-09-20)
 
    
    \bibitem{Linux}
    農林水産研究情報総合センター : UNIX(Linux)でよく使うコマンド ,入手先 \textless https:\slash\slash{}itcweb.cc.affrc.go.jp\slash{}affrit/doku.php?id=faq\slash{}tips\slash{}unix-command\textgreater 
    
    \bibitem{Cowrie}
    Micheloosterhof:Cowrie,入手先 \textless 
    https:\slash\slash{}itcweb.cc.affrc.go.jp\slash{}affrit\slash{}doku.php?id=faq\slash{}tips/unix-command\textgreater
    
    \bibitem{Virus}
    Virus Total 入手先,
    \textless https:\slash\slash{}www.virustotal.com\slash{}home\slash{}upload\textgreater
\end{thebibliography}

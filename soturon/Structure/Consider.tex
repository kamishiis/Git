\chapter{評価実験}
本研究で実装した検知システムにおける定常的な動作負荷の評価を行い,DDoS攻撃を行うマルウェアを用いて検知精度の評価を行った.

\section{システムコール呼び出し履歴を用いた検知手法による定常的な動作負荷の評価}

Java言語との比較では,惨敗であり,FUNは2倍の
記述量を必要とした.しかし,これは,Javaのもつ
パッケージIKURAが非常に強力であるためで,
同一機能をもつライブラリを用意することにより,
FUNにも同様の能力を持たせることができることが判明した.

\section{Miraiとその亜種マルウェアを対象とする判別性能評価}
ハニーポットを用いてDDoS攻撃を行うマルウェアを収集し,収集したマルウェアを検体として用いて,システムコール呼び出し履歴を用いた検知手法の検知精度の評価を行った.

\subsection{ハニーポットによるマルウェアの収集}
ハニーポットと呼ばれる攻撃者に端末を意図的に侵入させ,マルウェアをダウンロードさせ実行するまでの挙動を取得するシステムを用いてDDoS攻撃を行うマルウェアを収集した.シェルの対話の中でダウンロードされるバイナリファイルを実行させることなく保存することが可能なMichel Oosterhofによって開発されたCowrieと呼ばれるハニーポットを用いた.Cowrieによって収集されたバイナリファイルについてVirus Totalと呼ばれるマルウェア検知オンラインサービスを用いて解析を行い,DDoS攻撃を行うマルウェアの分類分けを行った.Virus Totalはユーザーから投稿された検体を54のウィルス対策エンジンによって解析するオンラインサービスであり,投稿された検体についてマルウェアの分類を知ることができる.2019/01/09から2019/01/28の期間でハニーポットを断続的に運用してバイナリファイルの収集を行った.収集したバイナリファイルをVirus Totalに投稿し,Virus Totalの解析結果から,MiraiまたはMiraiの亜種のマルウェアをDDoS攻撃を行うマルウェアとして分類分けした.分析した結果,収集したバイナリファイルは,空ファイルのものや,マルエウェアではないバイナリファイル,DDoS攻撃を行うマルウェアのバイナリファイルが散見され,DDoS攻撃を行うマルウェアは〇〇検体が存在した.

\subsection{提案システムによるマルウェアの検知精度評価}
前項で収集したOO検体のDDoS攻撃を行うマルウェアを用いてシステムコール呼び出し履歴を用いた検知手法の検知精度の評価を行った.入手したマルウェアをネットワークから隔離した状態で検知システムを動作させマルウェアの検知できるかどうか確認を行った.検知精度は以下の式で求めた.

\begin{equation}
    accuracy = 提案システムによるマルウェアの検知数/マルウェアの検体数
\end{equation}
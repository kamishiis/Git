\maketitle    % タイトルページを作成
%--------------------------------------------------------------------
% 英文概要(250語程度)
\begin{eabstract}
{In recent years, IoT devices equipped with communication functions in various things are spreading explosively. As a result It is a social problem that a botnet is constructed by an IoT device infected with malware and a DDoS attack is performed. Malware called Mirai is published on the web site, and many variants of Mirai are made. In this research, we aim to detect malware that performs DDoS attack on IoT device , and to detect unknown malware. we pay attention to the fact that the variants of malware were make from published malware on web site, we detect malware that performs DDoS attack by determining whether there is a specific function of the original malware}
\end{eabstract}

% 英文キーワード(5個程度をコンマ(,)で区切って羅列する)
\begin{ekeyword}
DDoS attack, IoT Device, malware, Mirai, Linux
\end{ekeyword}

%--------------------------------------------------------------------
% 和文概要(400字程度)
\begin{jabstract}
近年,世の中にある様々なものに通信機能を搭載したIoT機器が爆発的に普及している.その結果,マルウェアに感染したIoT機器によってボットネットが構築され大規模なDDoS攻撃が行われ大きな問題となっている.その中でも,Miraiと呼ばれるマルウェアがWeb上で公開され,Miraiの亜種が多く作られている.本研究では,IoTデバイス本体においてDDoS攻撃を行うマルウェアを検知する手法を検討することによって,未知のマルウェアでも検知を行えることを目的とする.公開されているマルウェアを元に亜種が作成されていることに着目をして,オリジナルのマルウェアが持つ特定の関数の挙動を検出することによってDDoS攻撃を行うマルウェアの検知を行う.
\end{jabstract}

% 和文キーワード(5個程度をコンマ(,)で区切って羅列する)
\begin{jkeyword}
DDoS攻撃,IoTデバイス,マルウェア,Mirai,Linux
\end{jkeyword}
\tableofcontents % 目次を作成

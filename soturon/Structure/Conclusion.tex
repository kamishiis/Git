\chapter{結論}

\section{まとめ}

シンボルテーブルを用いた検知手法では,マルウェアの実行形式ファイルに含まれるシンボルテーブルを削除された場合には検知をすることができなかった.しかし,システムコール呼び出し履歴を用いた検知手法では,シンボルテーブルを削除された場合でもMiraiを検知することが可能だった.

\section{今後の展望}
スキャン活動を行っていないbashliteと呼ばれるDDoS攻撃を行うマルウェアは検知を行う事ができないため.スキャン活動の他にも検知条件を定めてスキャン活動を行っていないマルウェアも検知できるようにする必要がある.ハニーポットを用いて集めたマルウェアはMiraiが主だったため,Mirai亜種であるマルウェアの代表例である,Wicked,Owari,Satori,Hajimeのマルウェア検体を入手することができなかったため,ハニーポットでその検体を入手してMirai亜種のマルウェアの検知精度を求める必要がある.

\chapter{結論}

\section{まとめ}
本研究ではIoTデバイスの動的解析によるマルウェア検知手法としてMiraiが持つ関数によって呼び出されるシステムコール呼び出し履歴に着目することで,マルウェアがIoTデバイス上で動作していることを検出する手法を提案した。提案システムによるIoTデバイスの使用される資源はClamAVと比較し,メモリ使用率は○\%減,CPU使用率は○\%減であった.ハニーポットによって収集したDDoS攻撃を行うマルウェアを用いてマルウェアの検知精度を評価した結果,Accuracyは91.4\%,FPRの値が0\%と高い精度を示していた.しかし,Miriaのスキャンするportが変更されていた場合には,マルウェアの検出をすることができなかった.
本提案手法では,IoTデバイス上でマルウェアがC\&CサーバからのDoS攻撃命令を待機している状態でも,マルウェアの検出が行えるため,ホスト上でマルウェア検出するのに有効であると考えられる.

\section{今後の展望}
今後の展望として,本提案手法ではマルウェアの検出できないスキャン活動を行っていないbashliteと呼ばれるDDoS攻撃を行うマルウェアは検知を行う事ができないため.マルウェアのスキャン活動ではない挙動に着目をし新たな検知条件を定めスキャン活動を行っていないマルウェアも検知できるようにする必要がある.スキャン活動に着目した検知条件でも,ハニーポットを用いて収集されたMiraiの亜種マルウェアはスキャンしているportが23だけではなく他のportをスキャンしていることが確認されたため,マルウェアの侵入経路を明らかにし様々なportに対してスキャン活動が行われた場合でも,マルウェアの検出ができるよう検知条件を定める必要がある.
 本提案手法による不正な挙動の検出から逃れるために,攻撃者側がIoTデバイスにTelnetログインが成功にし,マルウェアをデバイス上にダウンロードさせる際に,killコマンドを用いて提案システムを停止させてからマルウェアの実行,ホワイトリストの改ざんによって,マルウェアの挙動を提案システムによって監視することができず,マルウェアの検出ができないことが想定される.しかし,検知動作の強制終了やホワイトリストが改ざんされる挙動を検出することがでできれば,スキャン活動を行っていないマルウェアでも検出することができると考えられる.

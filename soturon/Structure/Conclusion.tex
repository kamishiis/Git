\chapter{結論}

\section{まとめ}
%シンボルテーブルの話を書く.
本研究では,IoTデバイスの動的解析によるマルウェア検知手法としてMiraiが持つ関数に着目し,シンボルテーブルを用いた検知手法,システムコール呼び出し履歴を用いた検知手法の2つを提案した.シンボルテーブルを用いた検知手法では,マルウェアの実行形式ファイルに含まれるシンボルテーブルを削除された場合にはマルウェアの検出できない制約がある.ハニーポットによって収集されたマルウェアのほとんどがシンボルテーブルから使用される関数名を取得することができなかっためマルウェアの検出ができなかった。ClamAVと実装したシンボルテーブルを用いた検知システムの比較を行ったところ,メモリ使用率は12.5\%減,CPU使用率は88\%減であった.
ClamAVとシステムコール呼び出し履歴を用いた検知システムの比較を行ったところ,メモリ使用率は48.4\%減,CPU使用率は97.6\%減であった.提案手法では,Clam AntiVirusよりもIoTデバイスの少ない計算資源によってマルウェア探索を行うことができる.ハニーポットによって収集したDDoS攻撃を行うマルウェアを用いてシステムコール呼び出し履歴を用いた検知手法のマルウェアの検知精度を評価した結果,Accuracyは91.4\%,FPRの値が0\%と高い精度を示していた.しかし,Miriaのスキャンするportが変更されていた場合には,マルウェアの検出をすることができなかった.\par
システムコール呼び出し履歴を用いた検知手法では,IoTデバイス上でマルウェアがC\&CサーバからのDoS攻撃命令を待機している状態でも,マルウェアの検出が行えるため,ホスト上でマルウェア検出するのに有効であると考えられる.

\section{今後の展望}
今後の展望として,本提案手法ではマルウェアの検出できないスキャン活動を行っていないbashliteと呼ばれるDDoS攻撃を行うマルウェアは検知を行う事ができないため.マルウェアのスキャン活動ではない挙動に着目をし新たな検知条件を定めスキャン活動を行っていないマルウェアも検知できるようにする必要がある.スキャン活動に着目した検知条件でも,ハニーポットを用いて収集されたMiraiの亜種マルウェアはスキャンしているportが23だけではなく他のportをスキャンしていることが確認されたため,マルウェアの侵入経路を明らかにし様々なportに対してスキャン活動が行われた場合でも,マルウェアの検出ができるよう検知条件を定める必要がある.
本提案手法による不正な挙動の検出から逃れるために,攻撃者側がIoTデバイスにTelnetログインが成功にし,マルウェアをデバイス上にダウンロードさせる際に,killコマンドを用いて提案システムを停止させてからマルウェアの実行,ホワイトリストの改ざんによって,マルウェアの挙動を提案システムによって監視することができず,マルウェアの検出ができないことが想定される.しかし,検知動作の強制終了やホワイトリストが改ざんされる挙動を検出することがでできれば,スキャン活動を行っていないマルウェアでも検出することができると考えられる.
